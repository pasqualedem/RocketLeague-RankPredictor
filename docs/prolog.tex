% --- ugly internals for language definition ---
%
\makeatletter

% initialisation of user macros
\newcommand\PrologPredicateStyle{}
\newcommand\PrologVarStyle{}
\newcommand\PrologAnonymVarStyle{}
\newcommand\PrologAtomStyle{}
\newcommand\PrologOtherStyle{}
\newcommand\PrologCommentStyle{}

% useful switches (to keep track of context)
\newif\ifpredicate@prolog@
\newif\ifwithinparens@prolog@

% save definition of underscore for test
\lst@SaveOutputDef{`_}\underscore@prolog

% local variables
\newcount\currentchar@prolog

\newcommand\@testChar@prolog%
{%
  % if we're in processing mode...
  \ifnum\lst@mode=\lst@Pmode%
    \detectTypeAndHighlight@prolog%
  \else
    % ... or within parentheses
    \ifwithinparens@prolog@%
      \detectTypeAndHighlight@prolog%
    \fi
  \fi
  % Some housekeeping...
  \global\predicate@prolog@false%
}

% helper macros
\newcommand\detectTypeAndHighlight@prolog
{%
  % First, assume that we have an atom.
  \def\lst@thestyle{\PrologAtomStyle}%
  % Test whether we have a predicate and modify the style accordingly.
  \ifpredicate@prolog@%
    \def\lst@thestyle{\PrologPredicateStyle}%
  \else
    % Test whether we have a predicate and modify the style accordingly.
    \expandafter\splitfirstchar@prolog\expandafter{\the\lst@token}%
    % Check whether the identifier starts by an underscore.
    \expandafter\ifx\@testChar@prolog\underscore@prolog%
      % Check whether the identifier is '_' (anonymous variable)
      \ifnum\lst@length=1%
        \let\lst@thestyle\PrologAnonymVarStyle%
      \else
        \let\lst@thestyle\PrologVarStyle%
      \fi
    \else
      % Check whether the identifier starts by a capital letter.
      \currentchar@prolog=65
      \loop
        \expandafter\ifnum\expandafter`\@testChar@prolog=\currentchar@prolog%
          \let\lst@thestyle\PrologVarStyle%
          \let\iterate\relax
        \fi
        \advance \currentchar@prolog by 1
        \unless\ifnum\currentchar@prolog>90
      \repeat
    \fi
  \fi
}
\newcommand\splitfirstchar@prolog{}
\def\splitfirstchar@prolog#1{\@splitfirstchar@prolog#1\relax}
\newcommand\@splitfirstchar@prolog{}
\def\@splitfirstchar@prolog#1#2\relax{\def\@testChar@prolog{#1}}

% helper macro for () delimiters
\def\beginlstdelim#1#2%
{%
  \def\endlstdelim{\PrologOtherStyle #2\egroup}%
  {\PrologOtherStyle #1}%
  \global\predicate@prolog@false%
  \withinparens@prolog@true%
  \bgroup\aftergroup\endlstdelim%
}

% language name
\newcommand\lang@prolog{Prolog-pretty}
% ``normalised'' language name
\expandafter\lst@NormedDef\expandafter\normlang@prolog%
  \expandafter{\lang@prolog}

% language definition
\expandafter\expandafter\expandafter\lstdefinelanguage\expandafter%
{\lang@prolog}
{%
  language            = Prolog,
  keywords            = {},      % reset all preset keywords
  showstringspaces    = false,
  alsoletter          = (,
  alsoother           = @$,
  moredelim           = **[is][\beginlstdelim{(}{)}]{(}{)},
  MoreSelectCharTable =
    \lst@DefSaveDef{`(}\opparen@prolog{\global\predicate@prolog@true\opparen@prolog},
}

% Hooking into listings to test each ``identifier''
\newcommand\@ddedToOutput@prolog\relax
\lst@AddToHook{Output}{\@ddedToOutput@prolog}

\lst@AddToHook{PreInit}
{%
  \ifx\lst@language\normlang@prolog%
    \let\@ddedToOutput@prolog\@testChar@prolog%
  \fi
}

\lst@AddToHook{DeInit}{\renewcommand\@ddedToOutput@prolog{}}

\makeatother
%
% --- end of ugly internals ---


% --- definition of a custom style similar to that of Pygments ---
% custom colors
%\definecolor{PrologPredicate}{RGB}{000,031,255}
\definecolor{PrologPredicate}{RGB}{111,84,28}
\definecolor{PrologVar}      {RGB}{024,021,125}
\definecolor{PrologAnonymVar}{RGB}{000,127,000}
\definecolor{PrologAtom}     {RGB}{186,032,032}
\definecolor{PrologComment}  {RGB}{063,128,127}
\definecolor{PrologOther}    {RGB}{000,000,000}

% redefinition of user macros for Prolog style
\renewcommand\PrologPredicateStyle{\color{PrologPredicate}}
\renewcommand\PrologVarStyle{\color{PrologVar}}
\renewcommand\PrologAnonymVarStyle{\color{PrologAnonymVar}}
\renewcommand\PrologAtomStyle{\color{PrologAtom}}
\renewcommand\PrologCommentStyle{\itshape\color{PrologComment}}
\renewcommand\PrologOtherStyle{\color{PrologOther}}

% custom style definition 
\lstdefinestyle{Prolog-pygsty}
{
  language     = Prolog-pretty,
  upquote      = true,
  stringstyle  = \PrologAtomStyle,
  commentstyle = \PrologCommentStyle,
  literate     =
    {:-}{{\PrologOtherStyle :-}}2
    {,}{{\PrologOtherStyle ,}}1
    {.}{{\PrologOtherStyle .}}1
}


% global settings
\lstset
{
  captionpos = below,
  frame      = single,
  columns    = fullflexible,
  basicstyle = \ttfamily,
  numbers=left
}

\newcommand{\inprolog}[1]{\lstinline[style=Prolog-pygsty]{#1}}